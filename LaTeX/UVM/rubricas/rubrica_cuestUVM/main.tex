\documentclass[11pt,article,landscape]{memoir}
% Copyright (C) 2013 Andrew Gainer-Dewar <andrew.gainer.dewar@gmail.com>
% This file may be distributed and/or modified under the
% conditions of the LaTeX Project Public License, either
% version 1.2 of this license or (at your option) any later
% version. The latest version of this license is in:
% http://www.latex-project.org/lppl.txt
% and version 1.2 or later is part of all distributions of[
% LaTeX version 1999/12/01 or later.

\usepackage{agd-rubric}
\usepackage[utf8]{inputenc}

%\rubriccourse{Principios de Programación}
%\rubriccourse{Electrónica}
\rubriccourse{Análisis y Diseño de Circuitos Eléctricos}
\rubricterm{Licenciatura Semestral Otoño 2018}
%\rubricthing{Hola mundo}
%\rubrictopprompt{FIME}

\begin{document}
\maketitle

\newpage
% The argument determines the number of score buckets
\begin{rubrictable}{4}
  % Add your \rubricdesc items in increasing order!
  \rubriccat{Portada (x1)}{
    \rubricdesc{Necesita mejorar}{
    Portada con pocos datos y mal estructurada.

    }
    \rubricdesc{Aprendiz}{
      La portada contiene la mayoría de los datos solicitados.
    }
    \rubricdesc{Bien}{
      La portada contiene la mayoría de los datos solicitados y posee un formato estructurado y limpio.
      
      
    }
    \rubricdesc{Distinguido}{
    Contiene nombre de la universidad y facultad; nombre, grupo, horario y salón de la clase, nombre del profesor,
    nombre y matrícula del alumno, título del trabajo y fecha de entrega.

    La portada tiene un formato bien estructurado y limpio.

    }
  }
  
  \rubriccat{Respuestas precisas (x2)}{
    \rubricdesc{Necesita mejorar}{
        Ninguna respuesta desarrolla el tema de forma precisa, responde de forma directa los cuestionamientos, ni da una idea clara del tema que se aborda.
    }
    \rubricdesc{Aprendiz}{
        Sólo pocas de las respuestas desarrollan el tema de forma precisa, responden de forma directa los cuestionamientos y dan una idea clara del tema que se aborda.
    }
    \rubricdesc{Bien}{
        Casi todas las respuestas desarrollan el tema de forma precisa, responden de forma directa los cuestionamientos y dan una idea clara del tema que se aborda.  
    }
    \rubricdesc{Distinguido}{
        Todas las respuestas desarrollan el tema de forma precisa, responden de forma directa los cuestionamientos y dan una idea clara del tema que se aborda.
    }
  }
  
 
  
  \rubriccat{Cuestionario completo (x1)}{
    \rubricdesc{Necesita mejorar}{
    Respondió menos del 70\% de las preguntas
    }
    \rubricdesc{Aprendiz}{
        Respondió por lo menos el 70\% de las preguntas.
    }
    \rubricdesc{Bien}{
        Respondió por lo menos el 80\% de las preguntas.
    }
    \rubricdesc{Distinguido}{
        Respondió todas las preguntas.
    }
  }

\end{rubrictable}

\begin{rubrictable}{4}
  % Add your \rubricdesc items in increasing order!
 
  \rubriccat{Ortografía, gramática y puntuación (x1)}{
    \rubricdesc{Necesita mejorar}{
        Numerosos errores de gramática, puntuación, mayúsculas y ortografía.
    }
    \rubricdesc{Aprendiz}{
        Varios errores de gramática, puntuación, mayúsculas y ortografía.
    }
    \rubricdesc{Bien}{
        Casi no hay errores de gramática, puntuación, mayúsculas y ortografía.
    }
    \rubricdesc{Distinguido}{
        No hay errores de gramática, puntuación, mayúsculas y ortografía.
    }
  }
  
  \rubriccat{Citas bibliográficas y confiabilidad (x2)}{
    \rubricdesc{Necesita mejorar}{
        Ausencia de citas o citas hechas en un formato incorrecto.
        
        Los trabajos citados son solo sitios de Internet.
    }
    \rubricdesc{Aprendiz}{
        Inconsistencias evidentes. Pocos trabajos citados y las citas no tienen el formato correcto.
        
        Se incluyen tres referencias de peso.
    }
    \rubricdesc{Bien}{
        Algunos trabajos citados (textuales y visuales) son hechos en el formato correcto.

        Se incluyen cuatro referencias de peso.
    }
    \rubricdesc{Distinguido}{
        Todos los trabajos citados (textuales y visuales) son hechos en el formato correcto y sin errores. Se recomienda el estilo IEEE.
        
        Se incluyen más de cinco referencias de peso (e.g. artículos de revistas de ciencia, libros, pero no más de dos sitios de Internet).
    }
  }
  
\end{rubrictable}

Notas:\\
\begin{itemize}
    \item Puntaje máximo: 21 puntos.
    \item Si la tarea no contiene portada o datos para reconocer quién la ha entregado, pierde su validez.
    \item La tarea se sube antes de la fecha límite a Schoology en formato PDF sin excepción alguna.
    \item El plagio es razón suficiente para invalidar la tarea. Hacerlo tendrá por penalización la invalidez de las tareas que se han entregado anteriormente. Consultar el siguiente enlace para más información: https://scielo.conicyt.cl/scielo.php?script=sci\_arttext\&pid=S0718-07642008000400001
\end{itemize}



\end{document}