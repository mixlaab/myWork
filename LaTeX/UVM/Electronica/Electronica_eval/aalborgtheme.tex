\documentclass[10pt]{beamer}
\usetheme[
%%% options passed to the outer theme
%    hidetitle,           % hide the (short) title in the sidebar
%    hideauthor,          % hide the (short) author in the sidebar
%    hideinstitute,       % hide the (short) institute in the bottom of the sidebar
%    shownavsym,          % show the navigation symbols
%    width=2cm,           % width of the sidebar (default is 2 cm)
%    hideothersubsections,% hide all subsections but the subsections in the current section
%    hideallsubsections,  % hide all subsections
%    left                % right of left position of sidebar (default is right)
  ]{Aalborg}
  
% If you want to change the colors of the various elements in the theme, edit and uncomment the following lines
% Change the bar and sidebar colors:
%\setbeamercolor{Aalborg}{fg=red!20,bg=red}
%\setbeamercolor{sidebar}{bg=red!20}
% Change the color of the structural elements:
%\setbeamercolor{structure}{fg=red}
% Change the frame title text color:
%\setbeamercolor{frametitle}{fg=blue}
% Change the normal text color background:
%\setbeamercolor{normal text}{bg=gray!10}
% ... and you can of course change a lot more - see the beamer user manual.

\usepackage[utf8]{inputenc}
%\usepackage[english]{babel}
\usepackage[spanish]{babel}
\usepackage[T1]{fontenc}
% Or whatever. Note that the encoding and the font should match. If T1
% does not look nice, try deleting the line with the fontenc.

\usepackage[table,xcdraw]{xcolor}
\usepackage{helvet}
\usepackage[spanish]{babel}
\usepackage{tikz}
\usetikzlibrary{shapes,arrows,positioning}

\usepackage{minted}


\usepackage{listings}
\usepackage{color}
\definecolor{codegreen}{rgb}{0,0.6,0}
\definecolor{codegray}{rgb}{0.5,0.5,0.5}
\definecolor{codepurple}{rgb}{0.58,0,0.82}
\definecolor{backcolour}{rgb}{0.95,0.95,0.92}
 
\lstdefinestyle{mystyle}{
    backgroundcolor=\color{backcolour},   
    commentstyle=\color{codegreen},
    keywordstyle=\color{magenta},
    numberstyle=\tiny\color{codegray},
    stringstyle=\color{codepurple},
    basicstyle=\footnotesize,
    breakatwhitespace=false,         
    breaklines=true,                 
    captionpos=b,                    
    keepspaces=true,                 
    numbers=left,                    
    numbersep=5pt,                  
    showspaces=false,                
    showstringspaces=false,
    showtabs=false,                  
    tabsize=2
}
 
\lstset{style=mystyle}


% colored hyperlinks
\newcommand{\chref}[2]{%
  \href{#1}{{\usebeamercolor[bg]{Aalborg}#2}}%
}

\title[Electrónica]% optional, use only with long paper titles
{Electrónica}

\subtitle{Evaluación}  % could also be a conference name

\date{\today}

\author[Víctor Medrano Zarazúa] % optional, use only with lots of authors
{
  Víctor Medrano Zarazúa\\
  \href{mailto:victor_medrano@my.uvm.edu.mx}{{\tt victor\_medrano@my.uvm.edu.mx}}
}
% - Give the names in the same order as they appear in the paper.
% - Use the \inst{?} command only if the authors have different
%   affiliation. See the beamer manual for an example

\institute[
%  {\includegraphics[scale=0.2]{aau_segl}}\\ %insert a company, department or university logo
  %Dept.\ of Electronic Systems\\
  Universidad del Valle de México\\
  Campus Monterrey
] % optional - is placed in the bottom of the sidebar on every slide
{% is placed on the bottom of the title page
  %Department of Electronic Systems\\
  Universidad del Valle de México\\
  Campus Monterrey
  %Universidad Autónoma de Nuevo León\\
  %Facultad de Ingeniería Mecánica y Eléctrica
  
  %there must be an empty line above this line - otherwise some unwanted space is added between the university and the country (I do not know why;( )
}

% specify the logo in the top right/left of the slide
\pgfdeclareimage[height=1cm]{mainlogo}{AAUgraphics/UVM} % placed in the upper left/right corner
\logo{\pgfuseimage{mainlogo}}

% specify a logo on the titlepage (you can specify additional logos an include them in 
% institute command below
\pgfdeclareimage[height=1.5cm]{titlepagelogo}{AAUgraphics/UVM} % placed on the title page
%\pgfdeclareimage[height=1.5cm]{titlepagelogo2}{AAUgraphics/aau_logo_new} % placed on the title page
\titlegraphic{% is placed on the bottom of the title page
  \pgfuseimage{titlepagelogo}
%  \hspace{1cm}\pgfuseimage{titlepagelogo2}
}

%\definecolor{UniBlue}{RGB}{255,255,255}

\tikzset{
block/.style={
  draw, 
  fill=blue!20, 
  rectangle, 
  minimum height=3em, 
  minimum width=6em
  },
 gain/.style={
    draw,
    fill=blue!20, 
    isosceles triangle,
    minimum height = 3em,
    isosceles triangle apex angle=60
    },
sum/.style={
  draw, 
  fill=blue!20, 
  circle, 
  },
input/.style={coordinate},
output/.style={coordinate},
pinstyle/.style={
  pin edge={to-,thin,black}
  }
}  

\begin{document}
% the titlepage


%\setbeamercolor{title}{fg=UniBlue}
%\setbeamercolor{normal text}{fg=UniBlue}
%\setbeamercolor{Aalborg}{fg=black,bg=black}


{\aauwavesbg
\begin{frame}[plain,noframenumbering] % the plain option removes the sidebar and header from the title page
  \titlepage
\end{frame}}
%%%%%%%%%%%%%%%%

% TOC
%\begin{frame}{Problemas}{}
%\tableofcontents
%\end{frame}
%%%%%%%%%%%%%%%%


\section{Evaluación}
\begin{frame}{Evaluación}{General}

\begin{columns}[c]
\column{1.7in}
\begin{block}{Primer Parcial}
\begin{itemize}
    \item Examen (40\%)
    \item Tareas (40\%)
    \item Participación (20\%)
\end{itemize}
\end{block}

\begin{block}{Segundo Parcial}
\begin{itemize}
    \item Examen (40\%)
    \item Tareas (40\%)
    \item Participación (20\%)
\end{itemize}
\end{block}

\column{1.7in}
\begin{block}{Tercer Parcial}
\begin{itemize}
    \item Proyecto (50\%)
    \item Examen (20\%)
    \item Tareas (20\%)
    \item Participación (10\%)
\end{itemize}
\end{block}


\end{columns}
\end{frame}

\section{Examen}
\begin{frame}{Examen}{Estructura}
\begin{block}{Teórico (50\%)}
    \begin{itemize}
        \item Abiertas
        \item De opción múltiple
    \end{itemize}
\end{block}

\begin{block}{Práctico (50\%)}
    \begin{itemize}
        \item Solución de problemas haciendo uso de análisis y operaciones matemáticas.
        \item Solución de problemas por medio de un algoritmo computacional.
    \end{itemize}
\end{block}

\end{frame}

\section{Tareas}
\begin{frame}{Tareas}{Aclaraciones}
\begin{block}{Uso de rúbricas}
    \begin{itemize}
        \item Investigación
        \item Cuestionario
        \item Algoritmo computacional
        \item Problemas
    \end{itemize}
\end{block}

\begin{block}{Entrega}
    \begin{itemize}
        \item Schoology
        \item Documentos en formato PDF
        %\item Portafolio en Github
        \item Una semana p/ entregar
    \end{itemize}
\end{block}

\end{frame}

\section{Participación}
\begin{frame}{Participación}{Reglamento}
\begin{block}{Reglas p/ participación}
    \begin{itemize}
        \item Participación voluntaria (+1)
        \item Estudiante pasa al azar (+1)
        \item Estudiante se niega a pasar (-1)
        \item Estudiante no asistió y le toca pasar (-3)
    \end{itemize}
\end{block}

\end{frame}

\section{Proyecto}
\begin{frame}{Proyecto}{Vehículo aéreo no tripulado}
\begin{block}{Requisitos generales del proyecto}
    \begin{itemize}
        \item Vehículo permanece estable durante el vuelo.
        \item Puede ser controlado manualmente o funcionar de manera autónoma.
        \item Está permitido el uso de tarjetas controladoras de vuelo.
        \item Estructura es diseñada en CAD (e.g. SolidWorks, OnShape, etc.)
        \item Se entrega un reporte con todos los detalles sobre el proyecto.
    \end{itemize}
\end{block}

\end{frame}

%%%%%%%%%%%%%%%%%%%%%%%%%%%%5
\section{Información de contacto}
% contact information
\begin{frame}{Feedback}{Información de contacto}
En caso de comentarios, sugerencias, preguntas o errores en las diapositivas no dudes en contactarme.
  \begin{center}
    \insertauthor\\
    \chref{https://mixlaab.github.io}{https://mixlaab.github.io}\\
    WA: 8119022700\\
    %9220 Aalborg Ø
  \end{center}
\end{frame}
%%%%%%%%%%%%%%%%

{\aauwavesbg%
\begin{frame}[plain,noframenumbering]%
  \finalpage{Fin}
\end{frame}}
%%%%%%%%%%%%%%%%

\end{document}
