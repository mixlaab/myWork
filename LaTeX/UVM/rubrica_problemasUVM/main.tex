\documentclass[11pt,article,landscape]{memoir}
% Copyright (C) 2013 Andrew Gainer-Dewar <andrew.gainer.dewar@gmail.com>
% This file may be distributed and/or modified under the
% conditions of the LaTeX Project Public License, either
% version 1.2 of this license or (at your option) any later
% version. The latest version of this license is in:
% http://www.latex-project.org/lppl.txt
% and version 1.2 or later is part of all distributions of[
% LaTeX version 1999/12/01 or later.

\usepackage{agd-rubric}
\usepackage[utf8]{inputenc}

%\rubriccourse{Principios de Programación}
%\rubriccourse{Electrónica}
\rubriccourse{Análisis y Diseño de Circuitos Eléctricos}
\rubricterm{Licenciatura Semestral Otoño 2018}
%\rubricthing{Hola mundo}
%\rubrictopprompt{FIME}

\begin{document}
\maketitle

\newpage
% The argument determines the number of score buckets
\begin{rubrictable}{4}
  % Add your \rubricdesc items in increasing order!
  \rubriccat{Portada}{
    \rubricdesc{Necesita mejorar}{
    Portada con pocos datos y mal estructurada.

    }
    \rubricdesc{Aprendiz}{
      La portada contiene la mayoría de los datos solicitados.
    }
    \rubricdesc{Bien}{
      La portada contiene la mayoría de los datos solicitados y posee un formato estructurado y limpio.
      
      
    }
    \rubricdesc{Distinguido}{
    Contiene nombre de la universidad y facultad; nombre, grupo, horario y salón de la clase, nombre del profesor,
nombre y matrícula del alumno, título del trabajo y fecha de entrega.

La portada tiene un formato bien estructurado y limpio.

    }
  }

  \rubriccat{Entendimiento del problema}{
    \rubricdesc{Necesita mejorar}{
        No entiende lo suficiente para empezar o hacer algún progreso.
    }
    \rubricdesc{Aprendiz}{
        Entiende lo suficiente para resolver parte del problema. 
    }
    \rubricdesc{Bien}{
        Entiende el problema.
    }
    \rubricdesc{Distinguido}{
        Identifica puntos clave antes de empezar el problema y además lo entiende.
    }
  }
  
  \rubriccat{Uso de datos e información}{
    \rubricdesc{Necesita mejorar}{
        Usa información que es inapropiada.
    }
    \rubricdesc{Aprendiz}{
        Usa correctamente parte de la información apropiada.
    }
    \rubricdesc{Bien}{
        Usa la información apropiada de manera correcta.
    }
    \rubricdesc{Distinguido}{
        Explica porque la información apropiada es esencial para llegar a la solución.
    }
    }
    
    
    \rubriccat{Métodos o procedimientos (x2)}{
    \rubricdesc{Necesita mejorar}{
        Aplica procedimientos inapropiados
    }
    \rubricdesc{Aprendiz}{
        Aplica algunos procedimientos apropiados.
    }
    \rubricdesc{Bien}{
        Aplica procedimientos completamente apropiados.
    }
    \rubricdesc{Distinguido}{
        Explica porque los procedimientos que utiliza son los adecuados.
    }
  }
  }

\end{rubrictable}

\begin{rubrictable}{4}
  % Add your \rubricdesc items in increasing order!
  
  \rubriccat{Uso de representaciones}{
    \rubricdesc{Necesita mejorar}{
        Utiliza una representación que proporciona poca o ninguna información significativa sobre el problema.
    }
    \rubricdesc{Aprendiz}{
        Utiliza una representación que aporta algo de información importante.
    }
    \rubricdesc{Bien}{
        Utiliza una representación que describe el problema y aporta información relevante.
    }
    \rubricdesc{Distinguido}{
        Utiliza una representación que describe claramente el problema y aporta información relevante.
    }
  }
  
  \rubriccat{Respuesta (x2)}{
    \rubricdesc{Necesita mejorar}{
        No hay respuesta o la respuesta es equivocada debido a errores de procedimiento.
    }
    \rubricdesc{Aprendiz}{
        Existen errores del siguiente tipo: a) error al anotar mal un número de algún paso anterior, b) error al hacer operaciones, c) se da una respuesta parcial a un problema con múltiples respuestas, d) no hay enunciado que exprese la respuesta final o e) se encierra algún resultado diferente al que pide el problema.\\
    }
    \rubricdesc{Bien}{
        Respuesta correcta.
    }
    \rubricdesc{Distinguido}{
        Se llega a la respuesta correcta y se expresa un regla general para resolver problemas de ese tipo.
    }
  }
  
  \rubriccat{Limpieza (x2)}{
    \rubricdesc{Necesita mejorar}{
        Numerosos borrones y rayaduras. Los apuntes son ilegibles o no se hace uso de lápiz/pluma con punta fina.
    }
    \rubricdesc{Aprendiz}{
          Algunos borrones y rayaduras. Algunos apuntes son ilegibles.
    }
    \rubricdesc{Bien}{
        No existen borrones o rayaduras. Los apuntes son legibles.
    }
    \rubricdesc{Distinguido}{
        No existen borrones o rayaduras. Los apuntes son legibles.\\
        
        Usa formas creativas para seguir la solución del problema de manera más efectiva (e.g: Uso de colores)
    }
  }
  
\end{rubrictable}

Notas:\\
\begin{itemize}
    \item Puntaje máximo: 24 puntos.
    \item Si la tarea no contiene portada o datos para reconocer quién la ha entregado, pierde su validez.
    \item Si la tarea es completamente ilegible para el maestro se invalida (ver Limpieza)
    \item Obligatorio: Los problemas se hacen en la libreta u hojas de máquina. Posteriormente se escanea o toman fotos de buena calidad y se anexan de forma ordenada (no anexar fotos giradas 90° o 180°) en un documento que tendrá formato PDF.
    \item La tarea se sube antes de la fecha límite a Schoology en formato PDF sin excepción alguna.
    \item El plagio es razón suficiente para invalidar la tarea. Hacerlo tendrá por penalización la invalidez de las tareas que se han entregado anteriormente. Consultar el siguiente enlace para más información: https://scielo.conicyt.cl/scielo.php?script=sci\_arttext\&pid=S0718-07642008000400001
\end{itemize}



\end{document}