\documentclass[11pt,article,landscape]{memoir}
% Copyright (C) 2013 Andrew Gainer-Dewar <andrew.gainer.dewar@gmail.com>
% This file may be distributed and/or modified under the
% conditions of the LaTeX Project Public License, either
% version 1.2 of this license or (at your option) any later
% version. The latest version of this license is in:
% http://www.latex-project.org/lppl.txt
% and version 1.2 or later is part of all distributions of[
% LaTeX version 1999/12/01 or later.

\usepackage{agd-rubric}
\usepackage[utf8]{inputenc}

%\rubriccourse{Principios de Programación}
%\rubriccourse{Electrónica}
\rubriccourse{Análisis y Diseño de Circuitos Eléctricos}
\rubricterm{Licenciatura Semestral Otoño 2018}
%\rubricthing{Hola mundo}
%\rubrictopprompt{FIME}

\begin{document}
\maketitle

\newpage
% The argument determines the number of score buckets
\begin{rubrictable}{4}
  % Add your \rubricdesc items in increasing order!
  \rubriccat{Portada (x1)}{
    \rubricdesc{Necesita mejorar}{
    Portada con pocos datos y mal estructurada.

    }
    \rubricdesc{Aprendiz}{
      La portada contiene la mayoría de los datos solicitados.
    }
    \rubricdesc{Bien}{
      La portada contiene la mayoría de los datos solicitados y posee un formato estructurado y limpio.
      
      
    }
    \rubricdesc{Distinguido}{
    Contiene nombre de la universidad y facultad; nombre, grupo, horario y salón de la clase, nombre del profesor,
nombre y matrícula del alumno, título del trabajo y fecha de entrega.

La portada tiene un formato bien estructurado y limpio.

    }
  }

  \rubriccat{Sección de título (x1)}{
    \rubricdesc{Necesita mejorar}{
        No se incluye sección de título
    }
    \rubricdesc{Aprendiz}{
        No contiene alguno de los datos solicitados. 
    }
    \rubricdesc{Bien}{
        Contiene nombre del alumno, universidad y facultad; dirección de la institución y correo electrónico del alumno.
    }
    \rubricdesc{Distinguido}{
        Contiene nombre del alumno, universidad y facultad; dirección de la institución y correo electrónico del alumno.
        
        Tiene un formato limpio, formal y organizado.
    }
  }
  
  \rubriccat{Resumen/ Abstract (x1)}{
    \rubricdesc{Necesita mejorar}{
        Incompleto o sin enfoque.
    }
    \rubricdesc{Aprendiz}{
        Enuncia el propósito de la investigación en una sola oración
    }
    \rubricdesc{Bien}{
        Enuncia clara y concisamente el propósito de la investigación en una sola oración.
    }
    \rubricdesc{Distinguido}{
        Enuncia clara y concisamente el propósito de la investigación en una sola oración que es atractiva y provoca leer acerca del tema.
    }
  }

\end{rubrictable}

\begin{rubrictable}{4}
  % Add your \rubricdesc items in increasing order!
  
  \rubriccat{Introducción (x2)}{
    \rubricdesc{Necesita mejorar}{
        No hay una introducción al tema de forma clara y tampoco presenta una estructura previa del trabajo de investigación.
    }
    \rubricdesc{Aprendiz}{
        La introducción enuncia el tema principal pero no da a conocer una estructura previa del trabajo de investigación.
    }
    \rubricdesc{Bien}{
        La introducción enuncia el tema principal y da a conocer cómo será la estructura del trabajo de investigación
    }
    \rubricdesc{Distinguido}{
        La introducción es atractiva, enuncia el tema principal, presenta antecedentes y da a conocer cómo será la estructura del trabajo de investigación.
    }
  }
  
  \rubriccat{Desarrollo (x2)}{
    \rubricdesc{Necesita mejorar}{
        Cada párrafo falla en desarrollar la idea principal.

        No hay evidencia de estructura o
        organización.
    }
    \rubricdesc{Aprendiz}{
        Cada párrafo tiene falta de oraciones detalladas de apoyo.
        
        Algunas ideas tienen un orden lógico pero su organización dentro del escrito no es coherente. Con frecuencia los párrafos son inconexos y no tienen sentido juntos
    }
    \rubricdesc{Bien}{
        Cada párrafo tiene oraciones detalladas de apoyo que desarrollan la idea principal.
        
        Las ideas a través del escrito están presentes pero no perfeccionadas al momento de hilarse.
    }
    \rubricdesc{Distinguido}{
        Cada párrafo tiene oraciones detalladas de apoyo que desarrollan la idea principal.
        
        El escrito demuestra un orden lógico y una secuencia sútil y coherente de las ideas a través de cada párrafo.
    }
  }
  
  \rubriccat{Conclusión (x2)}{
    \rubricdesc{Necesita mejorar}{
        La conclusión está incompleta y fuera de enfoque.
    }
    \rubricdesc{Aprendiz}{
        La conclusión no reafirma de manera adecuada lo que se mostró a lo largo del texto.
    }
    \rubricdesc{Bien}{
        La conclusión reafirma de forma general el texto y aporta una visión personal acerca del tema de investigación.
    }
    \rubricdesc{Distinguido}{
        La conclusión es atractiva, reafirma lo dicho en el texto y aporta una visión personal acerca del tema de investigación.
    }
  }
  
  \rubriccat{Ortografía, gramática y puntuación (x1)}{
    \rubricdesc{Necesita mejorar}{
        Numerosos errores de gramática, puntuación, mayúsculas y ortografía.
    }
    \rubricdesc{Aprendiz}{
        Varios errores de gramática, puntuación, mayúsculas y ortografía.
    }
    \rubricdesc{Bien}{
        Casi no hay errores de gramática, puntuación, mayúsculas y ortografía.
    }
    \rubricdesc{Distinguido}{
        No hay errores de gramática, puntuación, mayúsculas y ortografía.
    }
  }
  
\end{rubrictable}

\begin{rubrictable}{4}
  % Add your \rubricdesc items in increasing order!
  \rubriccat{Citas bibliográficas (x2)}{
    \rubricdesc{Necesita mejorar}{
        Ausencia de citas o citas hechas en un formato incorrecto.
        
        Los trabajos citados son solo sitios de Internet.
    }
    \rubricdesc{Aprendiz}{
        Inconsistencias evidentes. Pocos trabajos citados y las citas no tienen el formato correcto.
        
        Se incluyen tres referencias de peso.
    }
    \rubricdesc{Bien}{
        Algunos trabajos citados (textuales y visuales) son hechos en el formato correcto.

        Se incluyen cuatro referencias de peso.
    }
    \rubricdesc{Distinguido}{
        Todos los trabajos citados (textuales y visuales) son hechos en el formato correcto y sin errores. Se recomienda el estilo IEEE.
        
        Se incluyen más de cinco referencias de peso (e.g. artículos de revistas de ciencia, libros, pero no más de dos sitios de Internet).
    }
  }
\end{rubrictable}

Notas:\\
\begin{itemize}
    \item Puntaje máximo: 36 puntos.
    \item Si la tarea no contiene portada o datos para reconocer quién la ha entregado, pierde su validez.
    \item La tarea se sube antes de la fecha límite a Schoology en formato PDF sin excepción alguna.
    \item El plagio es razón suficiente para invalidar la tarea. Hacerlo tendrá por penalización la invalidez de las tareas que se han entregado anteriormente. Consultar el siguiente enlace para más información: https://scielo.conicyt.cl/scielo.php?script=sci\_arttext\&pid=S0718-07642008000400001
\end{itemize}



\end{document}